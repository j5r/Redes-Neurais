\documentclass[12pt,a4paper]{article}
%%%%%%%%%% pacotes
\usepackage{amsmath,amsfonts,amssymb}
\usepackage[portuguese]{babel}
\usepackage[dvipsnames]{xcolor}
\usepackage[left=2cm,right=2cm,top=2cm,bottom=2cm]{geometry}
\usepackage{
    txfonts,
    %cancel,
    hyperref,
    graphicx,
}

%%%%%%%%%% configs
\title{Algoritmo 1: Perceptron}
\author{Junior R. Ribeiro \\ \url{jrodrib@usp.br}}
\date{\today}
\hypersetup{
    colorlinks=true,
    linkcolor=blue,
    filecolor=magenta,      
    urlcolor=cyan,
}
\setlength{\parskip}{12pt}
\setlength{\parindent}{12pt}
\DeclareMathOperator{\sign}{sign}
\newcommand{\txt}[1]{\mbox{\texttt{#1}}}
%%%%%%%%%% comandos

\begin{document}

\maketitle

\section{Um Neurônio}
Versão do algoritmo Perceptron, com 1 neurônio artificial.

ENTRADA: 

\[
\Big\{x_j\in\mathbb{R}^m,\quad t_j\in\{-1,1\},\quad j=1,...,n\Big\}, \quad \epsilon>0, \quad 0<\eta\le 1, \quad \txt{maxit}=1e3.
\]

INÍCIO

$\left\{
\begin{aligned}
w&=0_{1\times m}
\\
E &= 1_{1\times n}
\\
j&=0
\\
\theta &= 0
\\
\txt{count}&=0
\end{aligned}
\right.
$

WHILE$\Big( \txt{count}<\txt{maxit}\Big)$\\[-24pt]
\begin{quote}
$\txt{count++}$\\
$\txt{j++}$\\
IF$(\txt{j==n+1})$\\[-22pt]
\begin{quote}
$\txt{break}$
\end{quote}
$E_j = t_j-\sign\Big(\big\langle[x_j,-1],\ \ [w,\theta]\big\rangle\Big)$\\
IF$(|E_j|<\epsilon)$\\[-24pt]
\begin{quote}
$\txt{continue}$
\end{quote}
$[w,\theta]=[w,\theta]+\eta E_j[x_j,-1]$\\
$\txt{j}=0$
%{\color{cyan}$\txt{j--}$ \hfill (você quer atualizar pesos até que o erro seja menor que a tolerância?)}\\


\end{quote}





\section{Multicamada}



\end{document}

